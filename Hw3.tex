\documentclass{homework}
\title{Homework 3}
\begin{document}
\maketitle

\begin{problem}
We recall the definition of divided differences for convenience:
\[f[x_0] = f(x_0), \quad f[x_0,\vec x,x_n] = \frac{f[\vec x,x_n] - f[x_0,\vec x]}{x_n - x_0}.\]
\begin{enumerate}[label=(\roman*)]
\item We argue by induction. For \(m=0\) this is obvious. Suppose this holds for \(m-1\), then we expand the definition:
\[\begin{aligned}
f[x_0,\vec x,x_m]
&= \frac{f[\vec x,x_m] - f[x_0,\vec x]}{x_m - x_0}\\
&= \frac{1}{x_m-x_0} \left[\sum_{i=1}^{m} \frac{f(x_i)}{\prod_{1\le j \le m}^{j\ne i}(x_i-x_j)} - \sum_{i=0}^{m-1} \frac{f(x_i)}{\prod_{0\le j \le m-1}^{j\ne i}(x_i-x_j)}\right]\\
&= \frac{1}{x_m-x_0} \left[\sum_{i=0}^{m} \frac{f(x_i)(x_i-x_0)}{\prod_{0\le j \le m}^{j\ne i}(x_i-x_j)} - \sum_{i=0}^{m} \frac{f(x_i) (x_i-x_m)}{\prod_{0\le j \le m}^{j\ne i}(x_i-x_j)}\right]\\
&= \frac{1}{x_m-x_0} \sum_{i=0}^m \frac{f(x_i) (x_m - x_0)}{\prod_{0\le j\le m}^{j\ne i} (x_i-x_j)}\\
&= \sum_{i=0}^m \frac{f(x_i)}{\prod_{0\le j\le m}^{j\ne i} (x_i-x_j)}.
\end{aligned}\]
\item Problem 1.(i) already gives a permutation-invariant way to express divided differences. The denominator is always the coordinate of this node minus all the other nodes, multiplied together.
\item We apply the definition, and by (ii) we reorder the nodes to put \(x_k\) at the beginning.\qed
\end{enumerate}
\renewcommand{\qed}{}
\end{problem}

\begin{problem} The Chebyshev nodes are
\(x_k = \cos \frac{(2k+1)\pi}{2n+2}\) for \(0 \le k \le n\).
\begin{enumerate}[label=(\roman*)]
\item There are \(N(a,b) = \lfloor \frac{(b+1)(n+1)}{2} \rfloor - \lceil \frac{(a+1)(n+1)}{2} \rceil\) points in the interval, by elementary considerations. Since \(\lfloor x \rfloor = x + O(1)\), we conclude that
\[\begin{aligned}
\lim_{n\to\infty} \frac{N(a,b)}{n+1}
&= \lim_{n\to\infty} \frac1{n+1} \left[\frac{(b+1)(n+1)}{2} - \frac{(a+1)(n+1)}{2} + O(1)\right]\\
&= \lim_{n\to\infty} \frac{b-a}{2} + O(1/n) = \frac{b-a}{2}.
\end{aligned}\]
\item We need to count the points with \[a \le \cos \frac{(2k+1)\pi}{2n+2} \le b.\] So we apply \(\arccos\) which is monotonic decreasing:
\[\arccos b \le \frac{2k+1}{2n+2} \pi \le \arccos a.\] So there are
\[N(a,b) = \left\lfloor \frac{n+1}{\pi}\arccos a -\frac 12\right\rfloor - \left\lceil \frac{n+1}{\pi}\arccos b -\frac 12\right\rceil.\]
By the same estimation we have
\[\lim_{n\to\infty}\frac{N(a,b)}{n+1} = \frac{\arccos a - \arccos b}{\pi}.\]
\item We take \(a = x\), \(b = x + \epsilon\). Since the fraction of points in the interval \([a,b]\) approaches \(\eta(a,b) = \frac{\arccos a - \arccos b}{\pi}\), we calculate the density
\[\lim_{\epsilon\to0}\frac{\eta(x,x+\epsilon)}{\epsilon} = -\frac1\pi \arccos' x = \frac{1}{\pi \sqrt{1-x^2}},\] as desired. \qed
\end{enumerate}
\renewcommand{\qed}{}
\end{problem}

\begin{problem} We have \(T_k(x) = \cos (k\arccos x)\).
\begin{enumerate}[label=(\roman*)]
\item The derivative of a Chebyshev polynomial is
\[T'_k(x) = \frac{k\sin(k\arccos x)}{\sqrt{1-x^2}}.\] Substituting the Chebyshev nodes, we have
\[T'_{n+1}(x_j) = \frac{(n+1)\sin \frac{ (2j+1)\pi}{2}}{\sin\frac{(2j+1)\pi}{2n+2}} = (-1)^{j}(n+1)\csc\frac{(2j+1)\pi}{2n+2}.\]
Therefore by trigonometric identities
\[T'_{n+1}(x_j) - T'_{n-1}(x_j) = \begin{cases}
2n(-1)^j & 1 \le j \le n-1\\
4n(-1)^j & j = 0 \lor j = n.
\end{cases}\]
\item By the Leibniz derivative rule we have
\[l'(x) = \sum_{i=0}^n \prod_{k\ne i} (x - x_k).\]
Substituting \(x = x_j\), we see that all terms except one contain \((x_j - x_j)\) and vanish. We are left with \(\prod_{k\ne j} (x_j - x_k)\). This shows \(\lambda_j = 1/l'(x_j)\).
\item For Chebyshev nodes, \(l(x) = 2^{1-n}T_n(x)\). Therefore using (ii) we only need to evaluate \(T'_n(x_j)\). We have the recurrence relation
\[T_{n+1}(x) = 2xT_n(x) - T_{n-1}(x),\]
Differentiating we have
\[T'_{n+1}(x) = 2T_n(x) + 2xT'_n(x) - T'_{n-1}(x).\]
Substituting \(x = x_j\) we see that \(T_n(x_j)\) vanishes. Recalling the results from (i) and after some elementary computation we are left with
\[T_n'(x_j) = (-1)^{j}n\]
and twice the result for \(j = 0, n\). Combining these we showed the desired equation.
\item \emph{On calcule!}
\[\begin{aligned}
\prod_{k\ne j} (x_j - x_k)
&= \prod_{k\ne j} \frac{2(j-k)}{n+1}\\
&= h^{n-1}\prod_{k \le j} (j-k) \cdot \prod_{k\ge j} (j-k)\\
&= (-1)^{n-j} h^{n-1} j! (n-j)!
\end{aligned}\]
which after taking the reciprocal, agrees with the result. \qed
\end{enumerate}
\renewcommand{\qed}{}
\end{problem}

\begin{problem}

\end{problem}

\begin{problem}
We have \(h = \frac2{n+1}\), plugging into the previous results we get
\[\frac{\lambda_{n/2}}{\lambda_n} = \frac{(-1)^{n/2}h^{n-1} (\frac n2)! (\frac n2)!}{(-1)^0 h^{n-1} n!} = (-1)^{n/2} {n \choose \frac n2}.\]
This ratio grows as \(O(4^n / \sqrt n)\), which is very fast. Therefore for a generic \(x\), the factor \(\lambda_{n/2}/\lambda_n\) will dominate the factor \((x - x_{n/2})/(x-x_n)\), thus by the second form of the barycentric formula, the polynomial will be much more sensitive to changes of the center datapoint compared to the endpoints.
\end{problem}

\end{document}
